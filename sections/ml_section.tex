% MIPT — Введение в финансы — Домашнее задание №2
% Файл: MIPT_Finance_HW1.tex
% Совместим с Overleaf (pdflatex)

\documentclass[12pt,a4paper]{article}

% --- кодировка и язык ---
\usepackage[T2A]{fontenc}
\usepackage[utf8]{inputenc}
\usepackage[russian]{babel}

% --- типография и шрифты ---
\usepackage{lmodern}
\usepackage{microtype}

% --- поля и оформление страницы ---
\usepackage{geometry}
\geometry{left=2.2cm, right=2.2cm, top=2.2cm, bottom=2.2cm}
\usepackage{setspace}
\onehalfspacing

% --- заголовки, колонтитулы ---
\usepackage{fancyhdr}
\pagestyle{fancy}
\fancyhf{}
\renewcommand{\headrulewidth}{0.4pt}
\renewcommand{\footrulewidth}{0.4pt}
\fancyhead[L]{МФТИ --- Введение в финансы}
\fancyhead[R]{Домашнее задание №5}
\fancyfoot[C]{\thepage}

% --- математика ---
\usepackage{amsmath,amssymb,amsthm,mathtools}
\allowdisplaybreaks

% --- другие полезные пакеты ---
\usepackage{siunitx}        % аккуратные числа и валюты
\usepackage{booktabs}       % красивые таблицы
\usepackage{graphicx}       % вставка рисунков
\usepackage{caption,subcaption}
\usepackage{tikz}           % простые схемы
\usepackage{enumitem}       % контроль списков
\usepackage{mdframed}       % рамки для решений
\usepackage[colorlinks=true,linkcolor=black,citecolor=black,urlcolor=black]{hyperref}
\usepackage{cleveref}

\title{Проект по Финансам}
\author{Стрелков Андрей}
\date{\today}

\begin{document}

\maketitle

\section{Описание системы и методов}

\subsection{Постановка задачи}

Рассматривается задача оптимальной остановки на случайном процессе цен. 
Дано ценовое траектория $X_0, X_1, \dots, X_{T}$, являющаяся марковским процессом. 
Требуется найти момент времени $\tau \le T$, максимизирующий доход:
\[
\text{profit}(\tau) = X_\tau - X_0.
\]
Оптимальная стратегия не известна. 

Цель: обучить модель, которая на каждом шаге времени решает, останавливать ли процесс. 
Это бинарная классификация:
\[
y_t = \begin{cases}
1, & \text{если } t = \text{оптимальная точка остановки},\\
0, & \text{иначе}.
\end{cases}
\]

\subsection{Генерация данных}

Для каждой траектории генерируется случайное блуждание длины $N$:
\[
X_{t+1} = X_t + \xi_t, \quad \xi_t = 
\begin{cases}
+1 & \text{с вероятностью } p, \\
-1 & \text{с вероятностью } 1-p,
\end{cases}
\]
где $p \sim U(0.4, 0.6)$ для каждой новой симуляции.

Таргет - $y_t = bool(X_t == max_{\tau \in [t, N]}(X_\tau))$

Также строятся верхняя и нижняя линейные границы методом наименьших квадратов:
\[
X_t \approx a t + b,
\]
\[
\text{upper}_t = a t + b + R_{\max} = a_ut+b_u, \qquad 
\text{lower}_t = a t + b + R_{\min} = a_lt+b_l,
\]
где $R_{\max}, R_{\min}$ — максимальные и минимальные остатки.

Каждый временной шаг кодируется шестью признаками:
\[
f_t = \{ 
X_t, N-t, a_u, b_u, a_l, b_l
\}
\]

Такая кодировка является общей для любого выбора границ в рамках этой задачи.
\subsection{Архитектура модели}

Используется однонаправленная рекуррентная нейросеть GRU:

\[
\text{GRU}_{\theta}: \mathbb{R}^{N \times d} \to \mathbb{R}^{N \times 1},
\]
где $d=6$ --- размерность входных признаков.

\begin{figure}[h]

\centering

\includegraphics[width=0.8\linewidth]{GRU.png}

\caption{GRU}

\label{fig:mpr}

\end{figure}

Формально:
\[
h_t = \text{GRUCell}(f_t, h_{t-1}), 
\qquad
\hat y_t = \sigma(W h_t + b).
\]

Параметры модели:
\begin{itemize}
    \item input size: $6$
    \item hidden size: $32$
    \item num layers: $1$
    \item activation на выходе: sigmoid
\end{itemize}

Выход модели — вероятности остановки:
\[
\hat y_t = \mathbb{P}(\text{stop at } t).
\]

Фактическая точка остановки определяется как
\[
\tau_{\text{model}} = 
\begin{cases}
\min\{t: \hat y_t \ge \text{threshold}\}, & \text{если есть срабатывание}, \\
N-1, & \text{иначе}.
\end{cases}
\]

\begin{figure}[h]

\centering

\includegraphics[width=0.8\linewidth]{загруженное (2).png}

\caption{Пример работы модели}

\label{fig:mpr}

\end{figure}

\subsection{Функция потерь}

Используется бинарная кросс-энтропия:
\[
\mathcal{L} = -\frac{1}{N}\sum_{t=0}^{N-1}
\left[
y_t \log(\hat y_t) + (1-y_t)\log(1-\hat y_t)
\right].
\]

Оптимизация: Adam.

\subsection{Метрики}

Оценка проводится на валидационном наборе.

\subsubsection{ROC-AUC}

Для всех временных шагов $t$ сравнивается распределение:
\[
(\hat y_t, y_t),
\]
и вычисляется ROC-AUC.

\begin{figure}[h]

\centering

\includegraphics[width=0.8\linewidth]{roc.png}

\caption{ROC}

\label{fig:mpr}

\end{figure}
\subsection{Прибыль модели}

Для каждой траектории $X_t$:
\[
\text{profit}_{\text{model}} = X_{\tau_{\text{model}}} - X_0.
\]

Максимальная возможная прибыль:
\[
\text{profit}_{\max} = \max_t X_t - X_0.
\]

Средняя прибыль на тесте:
\[
\mathbb{E}[\text{profit}].
\]

\begin{figure}[h]

\centering

\includegraphics[width=0.8\linewidth]{загруженное (1).png}

\caption{Прибыль модели}

\label{fig:mpr}

\end{figure}




\end{document}
