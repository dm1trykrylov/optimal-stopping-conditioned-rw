Для проведения численных экспериментов и проверки предложенного подхода необходимо разработать программный код,
реализующий связь между теоретической моделью и методами машинного обучения. 
На первом этапе программа должна обеспечивать синтез данных — генерацию множества траекторий случайного блуждания, 
ограниченных линейно изменяющимися по времени границами, задаваемыми инсайдерской информацией 
\( (a_1, b_1) \) и \( (a_2, b_2) \). Для каждой траектории определяется целевая величина — 
максимальная достигнутая стоимость актива, интерпретируемая как оптимальный момент продажи.
На основе полученных данных алгоритм должен принимать решение, исходя из текущего значения актива,
положения относительно границ и оставшегося времени до окончания периода (дедлайна продажи),
— продавать актив в данный момент или продолжать удержание. Желательно, чтобы программа также 
формировала оценку уверенности в своём решении, отражающую вероятность оптимальности выбранного действия. 
Такой функционал позволит использовать разработанный код для анализа стратегий поведения инвестора и
обучения моделей машинного обучения, приближающих оптимальное правило останова в условиях линейно изменяющихся ограничений.
