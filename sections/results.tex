%\begin{figure}[H]
%    \centering
%    \begin{subfigure}{0.4\textwidth}
%        \includegraphics[width=\textwidth]{./images/roc_N50.png}
%        \caption{$N=50$}
%    \end{subfigure}
%    \begin{subfigure}{0.4\textwidth}
%        \includegraphics[width=\textwidth]{./images/roc_N100.png}
%        \caption{$N=100$}
%    \end{subfigure}
%    \begin{subfigure}{0.4\textwidth}
%        \includegraphics[width=\textwidth]{./images/roc_N200.png}
%        \caption{$N=200$}
%    \end{subfigure}
%    \begin{subfigure}{0.4\textwidth}
%        \includegraphics[width=\textwidth]{./images/roc_N400.png}
%        \caption{$N=400$}
%    \end{subfigure}
%
%    \caption{ROC}
%    \label{fig:roc-curves}
%\end{figure}

Мы протестировали описанную выше модель и теоретические расчёты на сгенерированных данных.
Во всех датасетах было по 500 траекторий, а длина траектории ($N$) изменялась от 50 до 400.

\begin{figure}[H]

	\centering
	\begin{subfigure}{0.4\textwidth}
		\includegraphics[width=\textwidth]{./images/profit_vs_threshold_validation_N50.png}
		\caption{$N=50$}
	\end{subfigure}
	\begin{subfigure}{0.4\textwidth}
		\includegraphics[width=\textwidth]{./images/profit_vs_threshold_validation_N100.png}
		\caption{$N=100$}
	\end{subfigure}
	\begin{subfigure}{0.4\textwidth}
		\includegraphics[width=\textwidth]{./images/profit_vs_threshold_validation_N200.png}
		\caption{$N=200$}
	\end{subfigure}
	\begin{subfigure}{0.4\textwidth}
		\includegraphics[width=\textwidth]{./images/profit_vs_threshold_validation_N400.png}
		\caption{$N=400$}
	\end{subfigure}
	\caption{Прибыль модели и прибыль при использовании теоретических расчётов $m(n, k)$}

	\label{fig:profit}
\end{figure}

\begin{table}[H]
	\centering
	\begin{tabular}{lcccc}
		\toprule
		$N$ & model average & model median & theor average & oracle average \\
		\midrule
		100 & 6.68          & 6.00         & 6.55          & 7.96           \\
		200 & 7.99          & 6.00         & 7.83          & 10.36          \\
		400 & 11.51         & 10.00        & 11.00         & 14.80          \\
		\bottomrule
	\end{tabular}
	\caption{Прибыль модели при наилучшем пороговом значении (threshold) и прибыли при использовании теоретических расчётов $m(n, k)$.}
	\label{tab:profit-comparison}
\end{table}

По рис. \ref{fig:profit} видно, что оптимальное пороговое значение у модели сдвигается влево при росте $N$, но для $N \in \{100, 200, 400\}$ оптимальным оказалось значение 0.6.
Результаты для этого значения приведены в таблице \ref{tab:profit-comparison}. В последнем столбце в ней указано среднее из максимумов по каждой траектории.
В среднем модель показала лучшие результаты, чем теоретические расчёты.
Различие наименьшее при $N=100$ и достаточно существенно при $N=400$.

Преимущество ML-модели при небольших

\begin{figure}[H]

	\centering
	\includegraphics[width=\textwidth]{./images/sample_trajectories_N100.png}
	\caption{Примеры модельных и теоретических расчётов для конкретных траекторий, $N=100$}

	\label{fig:sample-trajectories-100}

\end{figure}

\begin{figure}[H]

	\centering
	\includegraphics[width=\textwidth]{./images/sample_trajectories_N200.png}
	\caption{Примеры модельных и теоретических расчётов для конкретных траекторий, $N=200$}

	\label{fig:sample-trajectories-200}

\end{figure}

\begin{figure}[H]

	\centering
	\includegraphics[width=\textwidth]{./images/sample_trajectories_N400.png}
	\caption{Примеры модельных и теоретических расчётов для конкретных траекторий, $N=400$}

	\label{fig:sample-trajectories-400}

\end{figure}


%N=100
%model_avg=6.68, model_med=6.0, theor_mean=6.546666622161865, oracle_mean=7.960000038146973
%N=200
%model_avg=7.986666666666666, model_med=6.0, theor_mean=7.826666831970215, oracle_mean=10.359999656677246
%N=400
%model_avg=11.506666666666666, model_med=10.0, theor_mean=11.0, oracle_mean=14.800000190734863