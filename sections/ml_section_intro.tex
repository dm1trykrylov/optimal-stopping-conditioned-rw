\subsubsection{Постановка задачи}

Рассматривается задача оптимальной остановки на случайном процессе цен.
Требуется найти момент времени $\tau \le T$, максимизирующий доход:
\[
	\text{profit}(\tau) = X_\tau - X_0.
\]
Оптимальная стратегия не известна.

Цель: обучить модель, которая в каждый момент времени решает, останавливать ли процесс.
Это бинарная классификация:
\[
	y_t = \begin{cases}
		1, & \text{если } t = \text{оптимальная точка остановки}, \\
		0, & \text{иначе}.
	\end{cases}
\]

\subsubsection{Генерация данных}

Для каждой траектории генерируется случайное блуждание длины $N$:
\[
	X_{t+1} = X_t + \xi_t, \quad \xi_t =
	\begin{cases}
		+1 & \text{с вероятностью } p,   \\
		-1 & \text{с вероятностью } 1-p,
	\end{cases}
\]
где $p = 0.5$ для каждой новой симуляции.

Таргет - $y_t = bool(X_t == max_{\tau \in [t, N]}(X_\tau))$

Также строятся верхняя и нижняя линейные границы методом наименьших квадратов:
\[
	X_t \approx a t + b,
\]
\[
	\text{upper}_t = a t + b + R_{\max} = a_ut+b_u, \qquad
	\text{lower}_t = a t + b + R_{\min} = a_lt+b_l,
\]
где $R_{\max}, R_{\min}$ — максимальные и минимальные остатки.

Каждый временной шаг кодируется шестью признаками:
\[
	f_t = \{
	X_t, N-t, a_u, b_u, a_l, b_l
	\}
\]

Такая кодировка является общей для любого выбора границ в рамках этой задачи.

\subsubsection{Архитектура модели}

Используется однонаправленная рекуррентная нейросеть GRU:
\[
	\text{GRU}_{\theta}: \mathbb{R}^{N \times d} \to \mathbb{R}^{N \times 1},
\]
где $d=6$ --- размерность входных признаков.

\begin{figure}[h]

	\centering

	\includegraphics[height=0.5\linewidth]{./images/GRU.png}

	\caption{GRU}

	\label{fig:gru}
\end{figure}

Формально:
\[
	h_t = \text{GRUCell}(f_t, h_{t-1}),
	\qquad
	\hat y_t = \sigma(W h_t + b).
\]

Параметры модели:
\begin{itemize}
	\item input size: $6$
	\item hidden size: $32$
	\item num layers: $1$
	\item activation на выходе: sigmoid
\end{itemize}

Выход модели — вероятности остановки:
\[
	\hat y_t = \mathbb{P}(\text{stop at } t).
\]

Фактическая точка остановки определяется как
\[
	\tau_{\text{model}} =
	\begin{cases}
		\min\{t: \hat y_t \ge \text{threshold}\}, & \text{если есть срабатывание}, \\
		N-1,                                      & \text{иначе}.
	\end{cases}
\]

\begin{figure}[H]

	\centering

	\includegraphics[width=0.9\linewidth]{./images/sample_trajectories_small_01_cropped.png}

	\caption{Пример работы модели для $N=50$}

	\label{fig:sample_trajectories_01}

\end{figure}

\subsection{Функция потерь}

Используется бинарная кросс-энтропия:
\[
	\mathcal{L} = -\frac{1}{N}\sum_{t=0}^{N-1}
	\left[
		y_t \log(\hat y_t) + (1-y_t)\log(1-\hat y_t)
		\right].
\]

Оптимизация: Adam.

\subsection{Метрики}

Оценка проводится на валидационном наборе.

\subsubsection{ROC-AUC}

Для всех временных шагов $t$ сравнивается распределение:
\[
	(\hat y_t, y_t),
\]
и вычисляется ROC-AUC.

\subsection{Прибыль модели}

Для каждой траектории $X_t$:
\[
	\text{profit}_{\text{model}} = X_{\tau_{\text{model}}} - X_0.
\]

Максимальная возможная прибыль:
\[
	\text{profit}_{\max} = \max_t X_t - X_0.
\]

Средняя прибыль на тесте:
\[
	\mathbb{E}[\text{profit}].
\]