Рассматриваемая задача является более общим случаем рассмотренной в части 2 работы \cite{Kulikov2017}.
Отличие заключается в типе ограничений: линейная функция вместо константы.
Используя обозначения, предложенные в этой работе, мы покажем, как результатыиз неё переносятся на нашу задачу.

Определим фунцию
\begin{equation}
	m(n, k) = \sup\limits_{n \leq \tau \leq N} \Expect{X_{\tau} \mid X_{n} = k},
\end{equation}
где $ n = 1,\ldots, N $. Если $ m(n, k) = k$, то данное состояние $(n, k)$, где $n$ --- момент времени, а $ X_n = k$ --- состояние случайного
процесса $X_n$, является решением задачи об оптимальной остановке.

\subsection{Вычисление $m(n, k)$}

Заметим, что $ \forall k \; m(N,k) = k$.
Также так как при вычислении этой функции используются только шаги от текущего до последнего, значение на $n$-м шаге можно выразить через значения на следующем. Для этого при нахождении супремума мы отделяем $X_n$ и преобразуем матожидание по определению:
\begin{align}
	 & m(n, k) = \sup\limits_{n \leq \tau \leq N} \Expect{X_{\tau} \mid X_{n} = k} = \max \left(k, \sup\limits_{n + 1 \leq \tau \leq N} \Expect{X_{\tau} \mid X_{n} = k}\right) \label{s01:eq:mnk_start} \\
	 & \Expect{X_{\tau} \mid X_{n} = k} = \sum_{p} p \cdot \Pr{X_{\tau} = p \mid X_{n} = k} \label{s01:eq:mnk_expect}                                                                                    \\
	 & \Pr{X_{\tau} = p \mid X_{n} = k} = \frac{\Pr{X_{\tau} = p ,\; X_{n} = k}}{\Pr{X_{n} = k}} = \frac{\Pr{X_{\tau} = p ,\; X_{n} = k ,\; X_{n+1} = k\pm1}}{\Pr{X_{n} = k}} =                          \\
	 & = \frac{\Pr{X_{\tau} = p \mid X_{n} = k,\; X_{n+1} = k+1} \cdot \Pr{X_{n} = k,\; X_{n+1} = k+1}}{\Pr{X_{n} = k}} +                                                                                \\
	 & + \frac{\Pr{X_{\tau} = p \mid X_{n} = k,\; X_{n-1} = k-1} \cdot \Pr{X_{n} = k,\; X_{n+1} = k-1}}{\Pr{X_{n} = k}} \label{s01:eq:sum_probs}                                                         \\
\end{align}
Удобно ввести обозначение для вероятности из значения $k$ на $n$-м шаге перейти выше, то есть в $k+1$ на следующем шаге: $ p(n,k) = \Pr{X_{n+1} = k+1 \mid X_{n} = k}$. По условию мы также получаем $ \Pr{X_{n+1} = k-1 \mid X_{n} = k} = (1 - p(n,k))$. После подстановки \eqref{s01:eq:sum_probs} в исходные выражения \eqref{s01:eq:mnk_expect} и \eqref{s01:eq:mnk_start} получается искомая зависимость:
\begin{align}
	 & \Expect{X_{\tau} \mid X_{n} = k} = \Expect{X_{\tau} \mid X_{n + 1} = k + 1, X_{n} = k} \cdot p(n, k) +                    \\
	 & + \Expect{X_{\tau} \mid X_{n + 1} = k - 1, X_{n} = k} \cdot (1 - p(n, k)) \Rightarrow                                     \\
	 & m(n, k) = \max \left(k,\; p(n, k) \cdot m(n + 1, k + 1) + (1 - p(n, k)) \cdot m(n + 1, k - 1)\right). \label{eq:opt-stop}
\end{align}

Вероятность перехода $p(n, k)$ также может быть вычислена рекурсивно, если выразить её через число траекторий $l(n, k)$, удовлетворяющих исходным условиям и проходящих через состояние $(n, k)$:
\begin{align}
	 & p(n, k) = \Pr{X_{n+1} = k+1 \mid X_{n} = k} = \frac{l(n + 1, k + 1)}{l(n, k)},                              \\
	 & l(n, k) = \begin{cases}
		             l(n + 1, k - 1),                   & a + x \cdot (n + 1) + 1 \geq k + 1 > a + x \cdot (n + 1)     \\
		             l(n + 1, k + 1),                   & -b - y \cdot (n + 1) - 1 \leq k - 1 < -b - y \cdot (n + 1)   \\
		             l(n + 1, k - 1) + l(n + 1, k + 1), & -b - y \cdot (k + 1) + 1 \leq k \leq a + x \cdot (k + 1) - 1 \\
		             0,                                 & \text{иначе}
	             \end{cases}
\end{align}

На рис. \ref{s01:fig:mnk_example} приведён пример вычисления значения $m(n,k)$ с помощью формул, полученных выше.
\begin{figure}[!ht]
	\centering
	\includegraphics[width=\linewidth]{./images/optstop_example_plot}
	\caption{Пример вычисления $m(n, k)$ для $N = 5$ и ограничений $ -2 \leq X_n \leq 3 + n$. Синей линией обозначена оптимальная граница остановки}
	\label{s01:fig:mnk_example}
\end{figure}

\subsection{Свойства $m(n, k)$}

В этом разделе будут доказаны аналоги теорем 2.4 и 2.5 из \cite{Kulikov2017}.
Ключевым отличием нашего процесса, не позволяющим сразу воспользоваться этими теоремами, является отсутствие симметрии.
Поэтому необходимо доказать свойства функций $p(n, k)$ и $l(n, k)$, на которые опираются теоремы, другими способами.

%\begin{lemma}\label{lem:recurrence}
%	Для любой пары $(n,k)$, достижимой из $0$ и удовлетворяющей ограничениям $L_n$, выполнено соотношение
%	\[
%		\ell(n,k)=\ell(n+1,k+1)+\ell(n+1,k-1),
%	\]
%	где слагаемое, не удовлетворяющее ограничениям, считается равным нулю.
%	Также при фиксированном $k$ функция $ n \mapsto \ell(n, k) $ нестрого убывает по $n$.
%\end{lemma}

\begin{lemma}\label{lemma:log-concavity}
	Для функции $l(n, k)$ выполнено неравенство $ l(n, k)^{2} \geq l(n, k - 2) \cdot l(n, k + 2) $, при этом если какой-то из входящих в него членов подразумевает выход за заданные границы, он считается равным нулю.
\end{lemma}
\begin{proof}
	Будем считать, что выход за границы не происходит. Докажем обратной индукцией по $n$.

	База индукции. Если $n = N$, то по определению $ l(N, k) = 1$ для любого $k$. Неравенство выполнено.

	Индукционный переход. Пусть неравенство верно для шага $n+1$.
	\begin{align}
		A_{j} := l(n + 1, j), \quad A_{j}^{2} \geq A_{j - 2} A_{j + 2}
	\end{align}
	Перейдём к шагу $n$:
	\begin{align}
		 & B_{k} := l(n, k) = l(n + 1, k - 1) + l(n + 1, k - 1) = A_{k - 1} + A_{k + 1}                                                                                                            \\
		 & B_{k - 2} = A_{k - 3} + A_{k - 1},\; B_{k + 2} = A_{k + 3} + A_{k + 1}                                                                                                                  \\
		 & B_{k}^{2} - B_{k - 2} \cdot B_{k + 2} = A_{k - 1}^{2} + 2 A_{k - 1} A_{k + 1} + A_{k + 1}^{2} - A_{k - 3} A_{k + 3} - A_{k - 3} A_{k + 1} - A_{k - 1} A_{k + 3} - A_{k - 1} A_{k + 1} = \\
		 & = \underbrace{A_{k - 1}^{2} - A_{k - 3} A_{k + 1}}_{T_1} + \underbrace{A_{k + 1}^{2} - A_{k - 1} A_{k + 3}}_{T_2} + \underbrace{A_{k - 1} A_{k + 1} - A_{k - 3} A_{k + 3}}_{T_3}
	\end{align}
	Слагаемые $T_1$ и $T_2$ неотрицательны по предположению индукции.
	\begin{align}
		 & \begin{cases}
			   A_{k + 1}^{2} \geq A_{k - 1} A_{k + 3} \\
			   A_{k - 1}^{2} \geq A_{k - 3} A_{k + 1} \\
		   \end{cases} \Rightarrow A_{k + 1}^{2} A_{k - 1}^{2} \geq A_{k - 1} A_{k + 3} A_{k - 3} A_{k + 1} \Rightarrow \\
		 & A_{k - 1} A_{k + 1} \geq A_{k - 3} A_{k + 3} \Rightarrow T_3 \geq 0
	\end{align}
	Третье слагаемое также неотрицательно.
	Значит, выполнено неравенство $ B_{k}^{2} - B_{k - 2} B_{k + 2} \geq 0 $ и индукционный переход доказан.
\end{proof}

\begin{lemma}\label{lemma:p_up-property}
	Функция $k \mapsto p(n, k)$ нестрого убывает по $k$ при фиксированном $n$.
\end{lemma}
\begin{proof}
	Сравним $p(n, k)$ и $p(n, k + 2)$:
	\begin{align}
		 & p(n, k) - p(n, k + 2) = \frac{l(n + 1, k + 1)}{l(n, k)} - \frac{l(n + 1, k + 3)}{l(n, k + 2)} =       \\
		 & = \frac{l(n + 1, k + 1) \cdot l(n, k + 2) - l(n + 1, k + 3) \cdot l(n, k)}{l(n, k) \cdot l(n, k + 2)}
	\end{align}
	Соотношение между $p(n, k)$ и $p(n, k + 2)$ определяется числителем.
	Используя обозначения из леммы~\ref{lemma:log-concavity} и равенства $ l(n, k + 2) = l(n + 1, k + 3) + l(n + 1, k + 1) $ и $ l(n, k) = l(n + 1, k + 1) + l(n + 1, k - 1) $, его можно преобразовать так:
	\begin{align}
		A_{k + 1} \left(A_{k + 3} + A_{k + 1}\right) - A_{k + 3} \left(A_{k + 1} + A_{k - 1}\right) = A_{k + 1}^{2} - A_{k - 1} A_{k + 3} = A_{k + 1}^{2} - A_{k + 1 - 2} A_{k + 1 + 2}.
	\end{align}
	Последнее выражение неотрицательно в силу леммы \ref{lemma:log-concavity}. Это завершает доказательство неравенства $ p(n, k) \geq p(n, k + 2)$.
\end{proof}

\begin{theorem}\label{theorem:monotone} % (Аналог т. 2.4)
	Функция $m(n,k)$ обладает следующими свойствами:
	\begin{enumerate}
		\item (Монотонность по $k$) При фиксированном $n$ $m(n, k)$ строго возрастает по $k$. В частности, $m(n, k) > m(n, k - 2)$.
		\item (Монотонность по $n$) При фиксированном $k$ $m(n, k)$ нестрого убывает по $n$. В частности, $m(n, k) \geq m(n + 2, k)$.
		\item $ m(n - 1, k + 1) \leq m(n, k) + 1 $.
		\item $ m(n, k + 2) \leq m(n, k) + 2 $.
	\end{enumerate}
\end{theorem}
\begin{proof} Верно доказательство теоремы 2.4 из \cite{Kulikov2017} со следующими уточнениями.

	1. Вместо случая $(k = a,\; k - 2 = -a)$ нужно рассмотреть $ (k = U_{n},\; k - 2 = L_{n})$.
	При наложенных ограничениях эти два равенства не могут быть выполнены одновременно.

	3. Нестрогое убывание функции $ k \mapsto p(n,k)$ доказано в лемме~\ref{lemma:p_up-property}.
\end{proof}

\begin{theorem}\label{theorem:optimal-stopping-threshold} % (Аналог т. 2.5)
	(Оптимальная граница остановки)
	\begin{enumerate}
		\item Если $m(n, k) = k$, то $m(n - 1, k + 1) = k + 1$, $m(n  - 2, k + 2) = k + 2$, $m(n, k + 2) = k + 2$.
		\item Если $m(n, k) > k$, то $m(n - 1, k - 1) > k - 1$, $m(n  - 2, k) > k$, $m(n, k - 2) > k - 2$.
		\item Оптимальная граница остановки для $ t = t_{0}, \ldots, N $ является $ N - L - t $, где $L = -L_{N} = b + N y$ и $t_{0} = (N - L - a)/(x + 1)$.
	\end{enumerate}
\end{theorem}
\begin{proof}
	Для первых двух пунктов справедливо доказательство из теоремы 2.5 в \cite{Kulikov2017}.
	В третьем пункте в т. 2.5 явно используется симметрия, поэтому здесь будет приведено другое доказательство.

	3. Обозначим $x(t) = N - L - t$. Из пункта 1. следует равенство $m(t, x(t)) = x(t)$ для $ t = t_{0}, \ldots, N $.
	Если граница $-L$ недостижима из-за чётности или из-за того, что $X_n$ принимает только целые значения, подразумевается, что $x(t)$ --- это ближайшее большее целое число к $-L$. В любом случае по определению $m(N, -L) = -L$.
	Значение $t_0$ определяется как точка пересечения прямой $ x(t) $ и верхней границы $ a + x \cdot t $.
	\begin{align}
		N - L - t_{0} = a + x \cdot t_{0} \Leftrightarrow t_{0}(x + 1) = N - L - a \Leftrightarrow t_{0} = \frac{N - L - a}{x + 1}
	\end{align}
	Если получается дробное решение или решение не той чётности, в качестве $t_{0}$ выбирается наибольшее подходящее значение, не превышающее изначального.

	Уже доказано, что оптимальная граница остановки не выше $x(t)$. Возможны два варианта.

	Первый вариант. Если $ \forall\; t \in \{t_0,\ldots, N\}$ между $x(t)$ и $L_{t}$ нет состояний, то оптимальная граница остановки не может быть ниже $x(t)$.

	Второй вариант. Пусть $ t_{1} $ --- наибольший номер, при котором состояние $x(t_{1}) - 2$ удовлетворяет ограничениям и достижимо.
	Условие на наибольший номер гарантирует, что из $(t_{1},\; x(t_{1}) - 2)$ можно пойти только вверх.
	Следовательно,
	\begin{align}
		 & m(t_{1}, x(t_{1}) - 2) = \max\left(x(t_{1}) - 2,\; m(t_{1} + 1, x(t_{1}) - 1)\right) =      \\
		 & = \max\left(x(t_{1} + 2),\; x(t_{1} + 1)\right) = x(t_{1} + 1) =  N - L - t_{1} - 1 \Rightarrow \\
		 & m(t_{1}, N - L - t_{1} - 2) = N - L - t_{1} - 1 > N - L - t_{1} - 2
	\end{align}
	Последнее неравенство вместе с неравенством из пункта 2 этой теоремы доказывает, что для всех $ t \leq t_{1} $ выполнено
	\begin{align}
		m(t, N - L - t - 2) > N - L - t - 2.
	\end{align}
	Таким образом, граница $x(t)$ дейстительно является оптимальной.
\end{proof}
