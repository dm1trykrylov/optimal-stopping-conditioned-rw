Теория оптимальной остановки случайных последовательностей --- одно из направлений теории случайных процессов и финансовой математики.
Классическая постановка и ряд классических задач теории случайных процессов были введены и рассмотрены в работе \cite{Shepp69}. 
В данной работе рассматривается задача оптимальной продажи актива при наличии инсайдерской информации.

Существует разные виды инсайдерской информации. 
Например, это может быть информация об объëме торгов на рынке или о продаже/покупке большого пакета активов. 
На основе такого рода информации можно сделать выводы о цене базового актива в конечный момент времени. 
Данная задача была исследована в работах \cite{Ekström2009, Kulikov2017}.

Также к инсайдерской информации мы относим известные верхнюю и нижнюю границы цен актива, устанавливаемые регулятором или полученные инвестором из других соображений, действующие весь период. 
В работе \cite{Kulikov2017} был изучен случай границ, не зависящих от времени. 
Мы рассматриваем более общий случай двух произвольных линейных по времени границ. 
Например, в результате анализа исторических данных инвестор может идентифицировать «уровни поддержки» и «уровни сопротивления», которые, как показано в \cite{Osler2000}, позволяют прогнозировать изменение цены в будущем.

В нашей работе цена актива моделируется с помощью симметричного случайного блуждания. 
Процесс, зажатый в рамках коридора, уже не является обычным случайным блужданием и имеет тренд, зависящий от состояния и момента времени. 
На случайное блуждание можно накладывать разные условия и анализировать его разными методами. 
В \cite{AHL_2019} рассматривается двумерное случайное блуждание, находящееся в единичном квадрате и не возвращающееся в начало. 
В \cite{Bertoin94} и \cite{Denisov2024} изучены асимптотические свойства одномерного случайного блуждания, принимающего только неотрицательные значения. 
Мы будем использовать подход, близкий к \cite{Denisov2024} и основанный на вероятностях перехода из текущего состояния.

В последние годы активно развивается направление, связанное с применением методов машинного обучения к задачам оптимальной остановки и финансового прогнозирования. 
Такие подходы позволяют строить аппроксимации оптимальных стратегий без необходимости аналитического решения сложных стохастических уравнений. 
В частности, в работах \cite{Becker2020, Chen2019} предложены методы на основе глубоких нейронных сетей и обучения с подкреплением (reinforcement learning), позволяющие агенту по историческим данным или сгенерированным траекториям случайного процесса обучаться принимать решение о моменте остановки, максимизирующем ожидаемую прибыль. 
Для задачи оптимальной продажи актива в условиях инсайдерской информации подобные модели могут использовать параметры текущего состояния (время, цену, положение относительно границ) в качестве признаков, а действие «продать» или «ждать» --- как результат обучения. 
Такой подход обеспечивает возможность численной аппроксимации оптимального правила останова даже в случаях, когда аналитическое решение отсутствует или сильно усложняется из-за наличия линейно изменяющихся границ.


