В данной работе рассматривается задача оптимальной продажи актива при наличии инсайдерской информации.

Существует разные виды инсайдерской информации. 
Например, это может быть информация об объëме торгов на рынке или о продаже/покупке большого пакета активов. 
На основе такого рода информации можно сделать выводы о цене базового актива в конечный момент времени. 
Данная задача была исследована в работах \cite{Ekström2009, Kulikov2017}.

Также к инсайдерской информации мы относим известные верхнюю и нижнюю границы цен актива, устанавливаемые регулятором или полученные инвестором из других соображений, действующие весь период. 
В работе \cite{Kulikov2017} был изучен случай границ, не зависящих от времени. 
Мы рассматриваем более общий случай двух произвольных линейных по времени границ. 
В нашей работе цена актива моделируется с помощью симметричного случайного блуждания. 

В последние годы активно развивается направление, связанное с применением методов машинного обучения к задачам оптимальной остановки и финансового прогнозирования. 
Такие подходы позволяют строить аппроксимации оптимальных стратегий без необходимости аналитического решения сложных стохастических уравнений. 
В частности, в работах \cite{Becker2020, Chen2019} предложены методы на основе глубоких нейронных сетей и обучения с подкреплением (reinforcement learning), позволяющие агенту по историческим данным или сгенерированным траекториям случайного процесса обучаться принимать решение о моменте остановки, максимизирующем ожидаемую прибыль.


