%%%%%%%%%%%%%%%%%%%%%%%%%%%%%%%%%%%%%%%%%
% Beamer Presentation
% LaTeX Template
% Version 2.0 (March 8, 2022)
%
% This template originates from:
% https://www.LaTeXTemplates.com
%
% Author:
% Vel (vel@latextemplates.com)
%
% License:
% CC BY-NC-SA 4.0 (https://creativecommons.org/licenses/by-nc-sa/4.0/)
%
%%%%%%%%%%%%%%%%%%%%%%%%%%%%%%%%%%%%%%%%%

%----------------------------------------------------------------------------------------
%	PACKAGES AND OTHER DOCUMENT CONFIGURATIONS
%----------------------------------------------------------------------------------------

\documentclass[
	11pt, % Set the default font size, options include: 8pt, 9pt, 10pt, 11pt, 12pt, 14pt, 17pt, 20pt
	%t, % Uncomment to vertically align all slide content to the top of the slide, rather than the default centered
	%aspectratio=169, % Uncomment to set the aspect ratio to a 16:9 ratio which matches the aspect ratio of 1080p and 4K screens and projectors
]{beamer}

\usepackage{slides-preamble}

%----------------------------------------------------------------------------------------
%	PRESENTATION INFORMATION
%----------------------------------------------------------------------------------------

\title[]{Задача об оптимальной остановке в случае наличия инсайдерской информации} % The short title in the optional parameter appears at the bottom of every slide, the full title in the main parameter is only on the title page

% \subtitle{Optional Subtitle} % Presentation subtitle, remove this command if a subtitle isn't required

\author[]{Дмитрий Крылов, Андрей Стрелков} % Presenter name(s), the optional parameter can contain a shortened version to appear on the bottom of every slide, while the main parameter will appear on the title slide

%\institute[МФТИ]{ФПМИ МФТИ \\ \smallskip \textit{krylov.de@phystech.edu}} % Your institution, the optional parameter can be used for the institution shorthand and will appear on the bottom of every slide after author names, while the required parameter is used on the title slide and can include your email address or additional information on separate lines

%\newdate{date}{13}{05}{2024}
%\date{\displaydate{date}}

%\date[13 мая 2024]{Мини-конференция программы <<Ментор>> \\ 13 мая 2024} % Presentation date or conference/meeting name, the optional parameter can contain a shortened version to appear on the bottom of every slide, while the required parameter value is output to the title slide

%----------------------------------------------------------------------------------------

\begin{document}

%----------------------------------------------------------------------------------------
%	TITLE SLIDE
%----------------------------------------------------------------------------------------

\begin{frame}
	\titlepage % Output the title slide, automatically created using the text entered in the PRESENTATION INFORMATION block above
\end{frame}

%----------------------------------------------------------------------------------------
%	TABLE OF CONTENTS SLIDE
%----------------------------------------------------------------------------------------

% The table of contents outputs the sections and subsections that appear in your presentation, specified with the standard \section and \subsection commands. You may either display all sections and subsections on one slide with \tableofcontents, or display each section at a time on subsequent slides with \tableofcontents[pausesections]. The latter is useful if you want to step through each section and mention what you will discuss.

%\begin{frame}
%	\frametitle{План} % Slide title, remove this command for no title
%
%	\tableofcontents % Output the table of contents (all sections on one slide)
%	%\tableofcontents[pausesections] % Output the table of contents (break sections up across separate slides)
%\end{frame}

%----------------------------------------------------------------------------------------
%	PRESENTATION BODY SLIDES
%----------------------------------------------------------------------------------------

%\section{Формулировка проблемы} % Sections are added in order to organize your presentation into discrete blocks, all sections and subsections are automatically output to the table of contents as an overview of the talk but NOT output in the presentation as separate slides

%\subsection{Цветовые пространства}

\begin{frame}
	\frametitle{Случайное блуждание}

	\begin{figure}
		\includegraphics[width=0.8\linewidth]{1d-unconstrained-walk.png}
		\caption{Симметричное случайное блуждание}
	\end{figure}
\end{frame}

\begin{frame}
	\frametitle{Случайное блуждание с ограничениями}

	\begin{figure}
		\includegraphics[width=0.8\linewidth]{1d-constrained-walk-large.png}
		\caption{Симметричное случайное блуждание с ограничениями}
	\end{figure}
\end{frame}

\begin{frame}
	\frametitle{Сравнение}

	\begin{figure}
		\includegraphics[width=0.8\linewidth]{1d-constrained-walk-distribution.png}
		\caption{Распределение значений в последнем периоде}
	\end{figure}
\end{frame}

\begin{frame}
	\frametitle{Сравнение}
	\begin{block}{Параметры}
		num\_periods = 10,\;
		num\_samples = 1000
	\end{block}

	\begin{table}
		\centering
		\begin{tabular}{l|c|c}
			     & unconstrained & constrained \\
			Mean & $0.12$        & $2.36$      \\
			std  & $3.05$        & $1.95$      \\
			Min  & $-10.0 $      & $0.0$       \\
			Max  & $8.0 $        & $10.0$      \\
		\end{tabular}
	\end{table}
\end{frame}

\begin{frame}
	\frametitle{Мотивация}

	\begin{itemize}
		\item От случайных блужданий можно перейти к винеровскому процессу (теорема Донскера-Прохорова).
		\item С помощью винеровского процесса можно моделировать стоимость некоторых активов (с высокой ликвидностью, без внешних шоков).
		\item Такой подход позволяет явно учесть инсайдерскую информацию.
		\item Есть исследования, в которых сравнивают предлагаемый подход с авторегрессионным, но без инсайдерской информации (\href{http://hdl.handle.net/10230/46217}{Cipolliti, 2020}).
	\end{itemize}

\end{frame}

\begin{frame}
	\frametitle{План}

	\begin{enumerate}
		\item Найти оптимальную границу остановки для симметричного случайного блуждания.
		\item Обучить ML-модель для решения задачи с инсайдерской информацией.
	\end{enumerate}

\end{frame}



%------------------------------------------------

%------------------------------------------------

%\subsection{Columns}
%
%\begin{frame}
%	\frametitle{Multiple Columns}
%	\framesubtitle{Subtitle} % Optional subtitle
%
%	\begin{columns}[c] % The "c" option specifies centered vertical alignment while the "t" option is used for top vertical alignment
%		\begin{column}{0.45\textwidth} % Left column width
%			\textbf{Heading}
%			\begin{enumerate}
%				\item Statement
%				\item Explanation
%				\item Example
%			\end{enumerate}
%		\end{column}
%		\begin{column}{0.5\textwidth} % Right column width
%			Lorem ipsum dolor sit amet, consectetur adipiscing elit. Integer lectus nisl, ultricies in feugiat rutrum, porttitor sit amet augue. Aliquam ut tortor mauris. Sed volutpat ante purus, quis accumsan dolor.
%		\end{column}
%	\end{columns}
%\end{frame}

%------------------------------------------------


%------------------------------------------------

%\subsection{Figure}
%
%\begin{frame}
%	\frametitle{Figure}
%
%	\begin{figure}
%		\includegraphics[width=0.8\linewidth]{creodocs_logo.pdf}
%		\caption{Creodocs logo.}
%	\end{figure}
%\end{frame}

%------------------------------------------------

%\section{Mathematics}
%
%\begin{frame}
%	\frametitle{Definitions \& Examples}
%
%	\begin{definition}
%		A \alert{prime number} is a number that has exactly two divisors.
%	\end{definition}
%
%	\smallskip % Vertical whitespace
%
%	\begin{example}
%		\begin{itemize}
%			\item 2 is prime (two divisors: 1 and 2).
%			\item 3 is prime (two divisors: 1 and 3).
%			\item 4 is not prime (\alert{three} divisors: 1, 2, and 4).
%		\end{itemize}
%	\end{example}
%
%	\smallskip % Vertical whitespace
%
%	You can also use the \texttt{theorem}, \texttt{lemma}, \texttt{proof} and \texttt{corollary} environments.
%\end{frame}

%------------------------------------------------
%
%\begin{frame}
%	\frametitle{Theorem, Corollary \& Proof}
%
%	\begin{theorem}[Mass--energy equivalence]
%		$E = mc^2$
%	\end{theorem}
%
%	\begin{corollary}
%		$x + y = y + x$
%	\end{corollary}
%
%	\begin{proof}
%		$\omega + F = \epsilon$
%	\end{proof}
%\end{frame}
%
%------------------------------------------------

%------------------------------------------------

%\begin{frame}[fragile] % Need to use the fragile option when verbatim is used in the slide
%	\frametitle{Verbatim}
%
%	\begin{example}[Theorem Slide Code]
%		\begin{verbatim}
%			\begin{frame}
%				\frametitle{Theorem}
%				\begin{theorem}[Mass--energy equivalence]
%					$E = mc^2$
%				\end{theorem}
%		\end{frame}\end{verbatim} % Must be on the same line
%	\end{example}
%\end{frame}

%------------------------------------------------

%------------------------------------------------

%\section{Referencing}
%
%\begin{frame}
%	\frametitle{Citing References}
%
%	An example of the \texttt{\textbackslash cite} command to cite within the presentation:
%
%	\bigskip % Vertical whitespace
%
%	This statement requires citation \cite{p1,p2}.
%\end{frame}
%
%%------------------------------------------------
%
%\begin{frame} % Use [allowframebreaks] to allow automatic splitting across slides if the content is too long
%	\frametitle{References}
%
%	\begin{thebibliography}{99} % Beamer does not support BibTeX so references must be inserted manually as below, you may need to use multiple columns and/or reduce the font size further if you have many references
%		\footnotesize % Reduce the font size in the bibliography
%
%		\bibitem[Smith, 2022]{p1}
%		John Smith (2022)
%		\newblock Publication title
%		\newblock \emph{Journal Name} 12(3), 45 -- 678.
%
%		\bibitem[Kennedy, 2023]{p2}
%		Annabelle Kennedy (2023)
%		\newblock Publication title
%		\newblock \emph{Journal Name} 12(3), 45 -- 678.
%	\end{thebibliography}
%\end{frame}

%----------------------------------------------------------------------------------------
%	ACKNOWLEDGMENTS SLIDE
%----------------------------------------------------------------------------------------

%----------------------------------------------------------------------------------------
%	CLOSING SLIDE
%----------------------------------------------------------------------------------------

%\begin{frame}[plain] % The optional argument 'plain' hides the headline and footline
%	\begin{center}
%		{\Huge The End}
%
%		\bigskip\bigskip % Vertical whitespace
%
%		{\LARGE Questions? Comments?}
%	\end{center}
%\end{frame}

%----------------------------------------------------------------------------------------

\end{document}
